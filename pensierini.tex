\documentclass[10pt,oneside]{amsproc}

\usepackage[utf8]{inputenc}
\usepackage[english]{babel}
\usepackage{comment}
\usepackage[alphabetic]{amsrefs}
\usepackage{tikz}
\usepackage{xcolor}
\usepackage{datetime2} 
\usepackage[colorlinks=true,linkcolor=blue,]{hyperref}

\usepackage[a4paper,margin=30mm]{geometry}
\usepackage{stackengine}
\usepackage{scalerel}
\usepackage{mathtools}
\usepackage{calc}
\usepackage{amsthm}
\usepackage{thmtools}
\usepackage[framemethod=TikZ]{mdframed}
\usepackage{amssymb}
\usepackage{amsfonts}
\usepackage{mathrsfs}
\usepackage{euscript}
% \def\EuScript#1{\mathscr#1}
% \usepackage{fourier-orns}
% \usepackage{palatino}
\usepackage{mathpazo} % add possibly `sc` and `osf` options
%\usepackage{eulervm}
%\usepackage{bbm}
%\usepackage{latexsym}
% \usepackage{mathrsfs}
%\usepackage{stmaryrd}
\usepackage{stix}
\usepackage{dsfont}
% \newcommand*{\TakeFourierOrnament}[1]{{%
% \fontencoding{U}\fontfamily{futs}\selectfont\char#1}}
% \renewcommand*{\danger}{\TakeFourierOrnament{66}}
\parindent0ex
\parskip1.2ex
\makeatletter
\DeclareOldFontCommand{\rm}{\normalfont\rmfamily}{\mathrm}
\DeclareOldFontCommand{\sf}{\normalfont\sffamily}{\mathsf}
\DeclareOldFontCommand{\tt}{\normalfont\ttfamily}{\mathtt}
\DeclareOldFontCommand{\bf}{\normalfont\bfseries}{\mathbf}
\DeclareOldFontCommand{\it}{\normalfont\itshape}{\mathit}
\DeclareOldFontCommand{\sl}{\normalfont\slshape}{\@nomath\sl}
\DeclareOldFontCommand{\sc}{\normalfont\scshape}{\@nomath\sc}
\makeatother

\newcommand{\mylabel}[1]{{#1}\hfill}
\renewenvironment{itemize}
  {\begin{list}{$\cdot$}{%
  \setlength{\parskip}{0mm}
  \setlength{\topsep}{.4\baselineskip}
  \setlength{\rightmargin}{0mm}
  \setlength{\listparindent}{0mm}
  \setlength{\itemindent}{0mm}
  \setlength{\labelwidth}{3ex}
  \setlength{\itemsep}{.2\baselineskip}
  \setlength{\parsep}{.2\baselineskip}
  \setlength{\partopsep}{0mm}
  \setlength{\labelsep}{1ex}
  \setlength{\leftmargin}{\labelwidth+\labelsep}
  \let\makelabel\mylabel}}{%
\end{list}}

\declaretheoremstyle[
  headfont=\normalfont\bfseries,
  notefont=\bfseries,
  notebraces={(}{)},
  bodyfont=\normalfont,
  postheadspace=1em,
  mdframed={
  outerlinewidth=1pt,
  linecolor=gray!20,
  roundcorner = 1ex,    
  backgroundcolor=gray!10, 
  innerleftmargin=1ex,
  leftmargin=-1ex,
  innerrightmargin=1ex,
  rightmargin=-1ex,
  innertopmargin=1.5ex, 
  innerbottommargin=1ex, 
  skipabove=3ex,
  skipbelow=1ex}, 
]{mystyle}

\declaretheorem[style=mystyle]%
{theorem}
\declaretheorem[style=mystyle,sibling=theorem]%
{lemma,proposition,fact,corollary,definition,remark,example,claim,question}

\let\proof\relax
\declaretheoremstyle[
  spaceabove=6pt, 
  spacebelow=6pt, 
  headfont=\normalfont\itshape, 
  bodyfont = \normalfont,
  postheadspace=1em, 
  qed=\qedsymbol, 
  headpunct={.}]
{myproof} 
\declaretheorem[style=myproof, unnumbered]{proof}

\renewcommand*{\emph}[1]{%
   \smash{\tikz[baseline]\node[rectangle, fill=teal!25, rounded corners, inner xsep=0.5ex, inner ysep=0.2ex, anchor=base, minimum height = 2.7ex]{\strut #1};}}



%%%%%%% GETCOMMIT
\newcommand\dotGitHEAD{}
\newcommand\branch{}
\newcommand\commit{}


\makeatletter\let\myfilehandle\@inputcheck\makeatother

\openin\myfilehandle=.git/HEAD\relax

\begingroup\endlinechar-1
  \global\read\myfilehandle to \dotGitHEAD
\endgroup
\closein\myfilehandle

\newcommand\GetBranch{}
\def\GetBranch ref: refs/heads/#1\relax{\renewcommand{\branch}{#1}}

\expandafter\GetBranch\dotGitHEAD\relax

\makeatother
%%%%%%%%%%%%%%

\linespread{1.25}
\title{S1-prime types}

\author{C. L. C. L. Polymath}
\begin{document}

\hfill\texttt{Branch:\ \branch\ \ \DTMnow}\bigskip
\maketitle
\raggedbottom

\newcommand\questionsign[1][2ex]{%
  \renewcommand\stacktype{L}%
  \scaleto{\stackon[-.6pt]{\color{red}$\triangle$}{\color{red}\bfseries\small ?}}{#1}%
}

\newcommand\dangersign[1][2ex]{%
  \renewcommand\stacktype{L}%
  \scaleto{\stackon[-.6pt]{\color{red}$\triangle$}{\color{red}\bfseries\small !}}{#1}%
}
\def\equivL{\stackrel{\smash{\scalebox{.5}{\rm L}}}{\equiv}}


%%%%%%%%%%%%%%%%%%%%%%%%%%%%%%%%%%%
%%%%%%%%%%%%%%%%%%%%%%%%%%%%%%%%%%%
%%%%%%%%%%%%%%%%%%%%%%%%%%%%%%%%%%%
%%%%%%%%%%%%%%%%%%%%%%%%%%%%%%%%%%%
%%%%%%%%%%%%%%%%%%%%%%%%%%%%%%%%%%%
\section{Introduction}\label{intro}

A type $p(x)\subseteq L({\EuScript U})$ is \emph{S1-prime\/} over $A$ if it is invariant over $A$ and for every $\varphi(x\,z)\in L(A)$ and every sequence $\langle a_i:i<\omega\rangle$ of $A$-indiscernibles if 

\def\ceq#1#2#3{\parbox[t]{20ex}{$\displaystyle #1$}\parbox[t]{6ex}{$\displaystyle\hfil #2$}{$\displaystyle #3$}}

\ceq{\hfill p(x)}{\rightarrow}{\varphi(x\,;a_0)\vee\varphi(x\,;a_1)}

then $p(x)\rightarrow\varphi(x\,;a_0)$.

Let \emph{${\rm cd}_A(x)$\/} be the type containing the formulas $\varphi(x\,;b)$, where $\varphi(x\,;z)\in L(A)$ and $b\in{\EuScript U}^{|z|}$, such that $\neg\varphi(x\,;b)$ divides over $A$.
That is, the type $\{\neg\varphi(x\,;b_i):i\in\omega\}$ is inconsistent for some $A$-indiscernible sequence $\bar b=\langle b_i:i\in\omega\rangle$, with $b_0\equiv_Ab$.

\begin{fact}
  The type ${\rm cd}_A(x)$ is consistent; invariant over $A$; and closed under logical consequences.
\end{fact}

\begin{proof}
  First, we verify that ${\rm cd}_A(x)$ is closed under conjunctions.
  Let $\bar b$ and $\bar c$ be two sequences of indiscernbles over $A$ such that $\big\{\neg\varphi(x\,;b_i):i\in\omega\big\}$ and $\big\{\neg\psi(x\,;c_i):i\in\omega\big\}$ are inconstistent.
  Let  $\langle b'_i,c'_i:i\in\omega\rangle$ be a sequence of indiscernibles with the same EM-type over $A$ as $\langle b_i,c_i:i\in\omega\rangle$.
  We claim that $\big\{\neg\big[\varphi(x\,;b'_i)\wedge\psi(x\,;c'_i)\big]:i\in\omega\big\}$ is inconsistent.
  In fact, otherwise at least one between $\big\{\neg\varphi(x\,;b'_i):i\in\omega\big\}$ and $\big\{\neg\psi(x\,;c'_i):i\in\omega\big\}$ would be consistent contradicting the requirement on the EM-type.

  Now, we verify that ${\rm cd}_A(x)$ is closed under logical consequences.
  In fact, if $\varphi(x\,;b)\rightarrow\psi(x\,;b)$ and $\neg\varphi(x\,;b)$ divides, then also $\neg\psi(x\,;b)$ divides.

  Finally, invariance over $A$ is clear by the definition and consistency follows from what proved above (as $\top$ does not divides).
\end{proof}

\begin{lemma} 
  The type ${\rm cd}_A(x)$ is S1-prime over $A$.
\end{lemma}

\begin{proof}
  Let $\langle a_i:i<\omega\rangle$ be a sequence of $A$-indiscernibles such that ${\rm cd}_A(x)\nrightarrow\varphi(x\,;a_0)$.
  By the definition of ${\rm cd}_A(x)$, the type $\big\{\neg\varphi(x\,;a_i):i<\omega\big\}$ is consistent.
  Then the sequence  $\langle a_{2i},a_{2i+1}:i<\omega\rangle$ witnesss that $\neg[\varphi(x\,;a_0)\vee\varphi(x\,;a_1)]$ does not divide over $A$.
  Then $\varphi(x\,;a_0)\vee\varphi(x\,;a_1)$ does not belongs to ${\rm cd}_A(x)$.
  By the fact above ${\rm cd}_A(x)\nrightarrow\varphi(x\,;a_0)\vee\varphi(x\,;a_1)$.
\end{proof}


\begin{lemma} 
  Let $p(x)$ be S1-prime over $A$.
  Then $p(x)\rightarrow{\rm cd}_A(x)$.
\end{lemma}

\begin{proof}
  It suffices to show that if $\neg\varphi(x, b)$ divides over $A$, then  $p(x)\rightarrow\varphi(x\,;b)$.
  Pick a sequence $\bar b$ of indiscernibles over $A$ such that $b_0\equiv_Ab$ and $\{\neg\varphi(x\,;b_i):i\in\omega)\}$ is inconsistent.
  Let $n$ be maximal such that $p(x)\nrightarrow\psi(x\,;b_{\restriction n})$, where
  
   \ceq{\hfill\psi(x\,;b_{\restriction n})}{=}{\bigvee_{i<n}\varphi(x\,;b_i)}
  
   If we can prove that $n=0$ then we conclude that $p(x)\rightarrow\varphi(x\,;b_0)$ and we are done by $A$-invariance.
   Suppose for a contradiction that $n>0$.
   Then, writing $I$ for $[n,2n)$, we have

  \ceq{\hfill p(x)}{\rightarrow}{\psi(x\,;b_{\restriction n})\vee\psi(x\,;b_{\restriction I}).}
 
  By S1-primality $p(x)\rightarrow\psi(x\,;b_{\restriction n})$ which contradicts the choice of $n$.
\end{proof}



% Let \emph{${\rm cnqi}_A(x)$\/} be the type containing the formulas $\varphi(x\,;b)$, where $\varphi(x\,;z)\in L(A)$ and $b\in{\EuScript U}^{|z|}$, such that $\neg\varphi(x\,;b)$ is not quasi-invariant over $A$.
% That is, the type $\{\neg\varphi(x\,;b_i):i\in\omega\}$ is inconsistent for some sequence $\bar b=\langle b_i:i\in\omega\rangle$, with $b_i\equiv_Ab$.

% \begin{fact}
%   The type ${\rm cnqi}_A(x)$ is consistent; invariant over $A$; and closed under logical consequences.
% \end{fact}


% \begin{proof}
%   First, we verify that ${\rm cnqi}_A(x)$ is closed under conjunctions.
%   Pick $\varphi(x\,;b)$ and $\psi(x\,;c)$
  
  
%   $\bar b$ and $\bar c$ be two sequences as above and such that $\big\{\neg\varphi(x\,;b_i):i\in\omega\big\}$ and $\big\{\neg\psi(x\,;c_i):i\in\omega\big\}$ are inconstistent.
%   Let  $c'=\langle b'_i,c'_i:i\in\omega\rangle$ be a sequence of indiscernibles with the same EM-type over $A$ as $\langle b_i,c_i:i\in\omega\rangle$.
%   We claim that $\big\{\neg[\varphi(x\,;b'_i)\wedge\varphi(x\,;c'_i)]:i\in\omega\big\}$ is inconsistent.
%   In fact, otherwise at least one between $\big\{\neg\varphi(x\,;b'_i):i\in\omega\big\}$ and $\big\{\neg\psi(x\,;c'_i):i\in\omega\big\}$ would be consistent contradicting the requirement on the EM-type.

%   Now, we verify that ${\rm cd}_A(x)$ is closed under logical consequences.
%   In fact, if $\varphi(x\,;b)\rightarrow\psi(x\,;b)$ and $\neg\varphi(x\,;b)$ divides, then also $\neg\psi(x\,;b)$ divides.

%   Finally, invariance over $A$ is clear by the definition and consistency follows from what proved above (as $\top$ does not divides).
% \end{proof}

We say that a relation $R(x\,;y)$ is stable if there is no infinite sequence $\langle a_i,b_i:i\in\omega\rangle$

\ceq{\hfill i<j}{\Leftrightarrow}{R(a_i\,;b_j)}

We say that $R(x\,;y)$ is an equation if there is no infinite sequence such that 

\ceq{\hfill i<j}{\Rightarrow}{R(a_i\,;b_j)}

\ceq{\hfill i=j}{\Rightarrow}{\llap{$\neg$} R(a_i\,;b_j)}

If $R$ is invariant over some set $A$, we can require that  $\langle a_i,b_i:i\in\omega\rangle$ is indiscernible.
  
\begin{fact}
  Let $p(x)\subseteq L({\EuScript U})$ be an $\omega_1$-prime $A$-invariant type.
  Let $\varphi(x,z)\in L(A)$.
  Then the relation 

  \ceq{\hfill R(a\,;b)}{\Leftrightarrow}{p(x)\rightarrow\varphi(x,a)\vee\varphi(x,b)}

  is equational.
\end{fact}

\begin{proof}
  Assume $R(a\,;b)$ is not equational.
  By compactness and invariance we can assume that there is an $A$-indiscernible sequence $\langle a_i,b_i:i\in\omega_1\rangle$ such that for every $i<j$

  \ceq{1.\hfill p(x)}{\rightarrow}{\varphi(x\,;a_i)\vee\varphi(x\,;b_j)}

  \ceq{2.\hfill p(x)}{\nrightarrow}{\varphi(x\,;a_i)\vee\varphi(x\,;b_i)}

  But $\omega_1$-primeness implies that $p(x)\rightarrow\varphi(x\,;a_i)$ for some $i$ which contradicts (2).
\end{proof}



A type is definable if for every $\varphi(x\,;z)\in L$ the set $\{a:p(x)\rightarrow\varphi(x\,;a)\}$ is type-definable.


%%%%%%%%%%%%%%%%%%%%%%%%%%%%%%%%%%%
%%%%%%%%%%%%%%%%%%%%%%%%%%%%%%%%%%%
%%%%%%%%%%%%%%%%%%%%%%%%%%%%%%%%%%%
%%%%%%%%%%%%%%%%%%%%%%%%%%%%%%%%%%%
%%%%%%%%%%%%%%%%%%%%%%%%%%%%%%%%%%%
\section{Pseudofinite random structures}\label{random}

Let $\Omega,\Pr$ be a finite probability space and let $M$ be a finite structure of signature $L_{\sf H}$.
Let $R= M^\Omega$ be the set of $M$-valued random variables.
We consider $R$ as a structure of signature $L_{\sf R}$ that contains the same function of $L_{\sf H}$.
Consider the two-sorted expansion $\langle M,R,I\rangle$ in the signature ${\EuScript L}$ that contains $L_{\sf H}$, $L_{\sf R}$, $L_{\sf I}$ and for every partitioned formula $\varphi(x\,;z)\in L_{\sf H}$ a function symbol of sort ${\sf R}^{|x|}\times{\sf H}^{|z|}\to I$.
Their interpretation is the function

\ceq{\hfill X,z}{\mapsto}{\Pr\varphi(X\,;z)}

Given an $L_{\sf H}$-theory $T$ with arbitrarily large finite models we define 

\ceq{\hfill T^{\sf R}}{=}{\big\{\varphi\in{\EuScript L}\ :\ \langle M,R,I\rangle\models\varphi\textrm{ for all sufficiently large }M,R\big\}}


\end{document}
